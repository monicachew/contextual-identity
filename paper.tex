\documentclass[10pt, conference, compsocconf]{IEEEtran}
%\usepackage[left=1in,top=1in,right=1in,bottom=1in,nohead]{geometry}

\usepackage{epsfig}
\usepackage{times}
\usepackage{latexsym}
\usepackage{graphicx}
\usepackage{ltxtable}
\usepackage{tabularx}
\usepackage{hhline}
\usepackage{color,colortbl}
\usepackage{url}
\usepackage{boxedminipage}
\usepackage{hyperref}
\usepackage{breakurl}
\usepackage{comment}
%\usepackage{fullpage}

\newcommand\T{\rule{0pt}{2.6ex}}
\newcommand\B{\rule[-1.2ex]{0pt}{0pt}}
\begin{document}

\title{Contextual Identity: Towards Usable Privacy}

\author{\IEEEauthorblockN{Monica Chew, Sid Stamm}
\IEEEauthorblockA{Mozilla\\
\tt{\{mmc, sid\}@mozilla.com}}
}
\maketitle

\begin{abstract}
We examine \textit{contextual identity}, a notion that individuals reveal
different aspects of themselves depending on context. At any given time,
an individual may act as a friend, relative, spouse, co-worker, aquaintance, or
stranger. We analyze contextual identity from the perspective of user choice
and control, survey contextual identity violations, and propose new resesarch
directions to enable users to have better control over their contextual
identities.
\end{abstract}

\section{Introduction}
\begin{quote}I am large, I contain multitudes. --- Walt Whitman,
\begin{em}Song of Myself\end{em} \end{quote}

The desire for privacy balances the desire for spontaneous, positive human
interaction, which often requires sharing personal information. The act of
sharing alone does not negate expectation of privacy~\cite{boyd2}.
%The desire for spontaneous, positive human interaction often requires sharing
%personal information. Sharing information can lead to loss of privacy, though
%it does not negate expectation of privacy~\cite{boyd2}. 
Helen Nissenbaum has long argued that privacy violations come not from the
simple sharing of personal information, but rather from sharing that
information in a way that violates social norms~\cite{nissenbaum}.

Lesbian, gay, bisexual, and transgendered youth are at risk for contextual
identity violations.  Access to support is crucial for LGTB-identified youth,
who are at higher risk for bullying and suicide~\cite{hrc}.
Much of that support is only accessible through the Internet in socially
conservative areas. In 2011 Bobbi Duncan and Taylor McCormick, two
students at the University of Texas at Austin, were inadvertantly outed when
their choir director added them to a Facebook group for Queer
Chorus~\cite{fowler}. Even though both students had blocked their parents from
seeing any personal posts revealing their sexuality, because Queer Chorus is a
public group Facebook published notifications to Duncan's and McCormick's
friends, including their parents. Duncan later attempted suicide~\cite{duncan}.

\begin{comment}
zuckerberg on "having multiple identities is dishonest"
http://books.google.ca/books?id=RRUkLhyGZVgC&pg=PA199&lpg=PA199&dq=%22The+days+of+you+having+a+different+image+for+your+work+friends+or+co-workers+and+for+the+other+people+you+know+are+probably+coming+to+an+end+pretty+quickly,%22&source=bl&ots=LRUZf_pd9m&sig=oHHi0qhlHh9prk94gCQFXItjGIg&hl=en&sa=X&ei=ma19UIOhAdOEqQHE2oCADA&ved=0CD8Q6AEwBA#v=onepage&q=%22The%20days%20of%20you%20having%20a%20different%20image%20for%20your%20work%20friends%20or%20co-workers%20and%20for%20the%20other%20people%20you%20know%20are%20probably%20coming%20to%20an%20end%20pretty%20quickly%2C%22&f=false
\end{comment}

In offline environments, managing contextual identities is more intuitive than
in online environments. There are cues to help you determine where you are,
who your intended audience is, how many people will overhear you, and how
likely your information is to be re-broadcast in a different context. 
In offline environments, humans don't have perfect memories and
will typically forget.  However, with the advent of increasingly vast
searching, indexing, and archiving capabilities, one cannot rely on forgetting
in online environments~\cite{delete}. Managing contextual identities
online has become increasingly difficult and fraught with mistakes; we must
provide users with better tools to control their contextual identities.

\section{Related Work}

The notion that individuals inhabit multiple personas has existed in multiple
fields for many years; Erving Goffman called this impression management, and
Carl Jung called it persona theory~\cite{goffman,jung}.

Barth et al.~provide a formal model for contextual
integrity of user data~\cite{barth}. Their work models
the flow of personal information in terms of knowledge states held by agents
acting in given roles, and works well for expressing privacy legislation such as
HIPPA and COPPA. Two limitations of this work are that it does not
model information where the subject involves more than one person, and it
only takes the type of information into account rather than the content or tone
of the information. For example, ``Bob told Alice that Charlie cooked a
delicious dinner'' is out of scope of the model.  In the context of social
networks, information is rarely about a single person, and the tone of the
message is as important to user effect as the contents.

Many authors have discovered methods to link different contextual identities to
an individual. Narayanan and Shmatikov present a re-identification technique
to merge anonymized social graphs from different networks and prove that
user identifiers from Netflix and IMDB and Flickr and Twitter can be
linked~\cite{narayanan1,narayanan2}. Lindamood et al.~and Mislove et al.~show
how to infer previously undisclosed information from public social networking
data~\cite{lindamood,mislove}.

Many authors have noted how re-broadcasting information information out of
context and making information discoverable can lead to user distress, even if
the information was previously public, but not easily
discoverable~\cite{boyd1,chew,nissenbaum}.

Evidence suggests that users' major privacy concerns come from people the user
knows: friends, family, and co-workers~\cite{fbtips2,fbtips1}.  Surveys
conducted by Wang et al.~suggest that typical regrets from posting on Facebook
stem almost exclusively from fear of negative interactions with people that the
user knows~\cite{wang}. The consequences of these negative interactions can
lead to loss of employment or breaking personal relationships.  Most previous
privacy research has focused on distant adversaries, where the bad actor is a
behavioral tracking service, state agent or unknown eavesdropper; in contrast,
we assert that users care more about their interactions with others close to
them.

\section{Contextual identity violations}
In this section we discuss three types of \textit{contextual identity
violations}. We consider a contextual identity violation to be when multiple
identities are linked without user intent, or when the user cannot choose which
identity to assert in a given context. In these examples the user
may not be conscious of what a contextual identity is: nevertheless each
example illustrates a loss of user control over what aspects of self to
reveal in a given context.

\label{sec:examples}
\subsection{Redistributing information out of context}
Redistributing information out of its original context often leads to
embarrassment~\cite{nissenbaum}.  In November 2007, Facebook Beacon allowed
third-party sites to publish purchases, travel bookings, movie rentals and more
to the user's activity stream.  Because of poor opt-out and lack of visibility
into what was being published, user outcry was
immediate~\cite{mccarthy,nakashima}.  In December 2007, Google Reader exposed
RSS feeds of user-marked news stories to the user's Google Talk contacts, which
includes everyone with whom a user has chatted whether they be co-workers,
supervisors, or friends. Although this feed was always public, prior to this
launch it was not discoverable, leading to a bad experience for many
users~\cite{helft}. In February 2010, Google Buzz launched, a product that
exposed the user's most frequent Gmail and Google Talk contacts and their
publicly available (through previously less discoverable) news and photos.  The
combination of exposing contacts (which could include abusive ex-husbands,
co-workers, and friends), as well as aggressively linking photos and news
items, resulted in a huge backlash~\cite{fugitivus,buzz}. In September 2012,
Facebook imported old wall posts from 2008 into the new Timeline interface.
Although wall posts were always visible from profile pages, the new Timeline
interface brought old wall posts, which users used to treat as private messages
before the advent of ``Like'' and comment buttons, to the attention of an
audience that the user never anticipated~\cite{timeline}.

\subsection{Policy errors}
\label{sec:policies}
Some service providers have policies which preclude isolating multiple
contextual identities. For example, Facebook and Google have a ``Real Names''
policy, which requires users to register for accounts with their legal
name.\footnote{Facebook's policy: \url{http://www.facebook.com/help/?page=258984010787183}, Google's policy: \url{http://support.google.com/plus/bin/answer.py?hl=en&answer=1228271}}
%\cite{fb_names,google_names}.  
These policies ignore that
community-building happens in many different contexts, that users have
legitimate reasons for presenting different identities in different contexts,
and that users don't necessarily want those identities to be linked. For
example, disallowing avatar handles as a primary identifier makes building a
gaming community difficult. % Linking a gaming identity to a professional
%identity has already caused problems for 2012 Maine Senate candidate and World
%of Warcraft gamer Colleen Lachowicz~\cite{maine}.
It is impossible to isolate multiple contextual identities in these networks
without violating the terms of service.

Even worse, many of these providers now serve as login platforms for external
sites. For example, Facebook Connect allows third-party websites to
authenticate users using their Facebook identity~\cite{fb_connect}.  Because it
is against the terms of service to have multiple Facebook accounts, using
Facebook Connect may have the unwanted side-effect of linking multiple
contextual identities across multiple sites.

Other federated login systems have support for multiple identifiers, but this
feature can be poorly implemented or difficult to discover, leading to low use.
Some login platforms and protocols, such as OAuth and BrowserID, do not suffer
from this policy or design error, giving a user better
control over which contextual identity to assert~\cite{browserid,oauth}.

\subsection{User interface errors}
It is all too easy for users to broadcast information to
an unintended audience. This phenomenon is so common on Twitter that it has its
own name, ``DM fail'', or Direct Message fail: when the user posts a public
message instead of a private, direct message. Representative Anthony Weiner was
a victim of this mistake when he inadvertently published compromising pictures
of himself~\cite{weiner}. Considering that this mistake requires
mistyping a single character (\texttt{@} instead of \texttt{d}), it's no
surprise that DM failures are so common.

Similar to DM failures, posting to the wrong account is also a common mistake.
Because many jobs require posting on social networks on behalf of the company,
it is common to have multiple accounts for personal and business use.
KitchenAid, Chrysler, and Google are three companies whose employees have
made this mistake in the past year~\cite{kitchenaid,chrysler,yegge}.

\section{Research directions}
We propose new research directions based on the following questions:
\begin{itemize}
\item How do users think about identity?
\item How do users manage identity?
\item How can we improve tools for managing identity?
\end{itemize}

\subsection{How do users think about identity?}
We hypothesize many users are not consciously aware of having multiple
contextual identities. In order to develop tools to help users, we need to
understand users' mental models of identity and how information is shared on
the Internet. For computer scientists, authentication is intrinsically linked
to identity, and even in the absence of authentication, using the same device
over time implicitly creates an identity through tracking techniques and local
information. However, these points are far from obvious to a typical user,
especially for users who don't distinguish the web application from the
browser, from the operating system, or from the device.\footnote{What is a
browser? \url{https://www.youtube.com/watch?v=o4MwTvtyrUQ}}

The computer science community must free itself from software-specific
notions of identity if we are to help users to whom the very concept of
identity is an enigma. To understand our users' mental models,
one might ask:

\begin{itemize}
\item How is identity represented online?
\item What do you need to represent your identity?
\item Is your identity tied to your device?
\item If you check your mail, read news, log on to a social network at a
computer at the public library, does that affect your identity?
\item What does it mean to share devices? If you lend your tablet to your
sister, is she representing herself, or you?
\item If you register for an account on a service, do you expect that to have
an influence on your online identity? What if you never use that account again?
\item If you visit a website, do you expect that to have an influence on your
browsing experience or online identity? If so, for how long?
\item Which parts of your online identity do you expect to be visible to
your housemates, friends, relatives, or employers?
\item Which parts of your online identity do you expect to be visible to
websites you use? What about websites you don't use?
\item Which parts of your identity do you want to share or keep secret, and from whom?
\item How does your identity change over time?
\end{itemize}

Pew Internet is a good source for phone survey data on privacy and social
media, and danah boyd's work on youth and social media is excellent, but there
is otherwise a dearth of research in these areas~\cite{boyd,pew1,pew2,pew3}.

\subsection{How do users manage identity?}

Users who are aware of having multiple identities engage in the following
techniques to separate, link, and curate identities.

\subsubsection{Separating identities}

The following examples illustrate how users keep their identities separate.

\paragraph{Multiple accounts}
The long-time existence of support for multiple accounts in email clients
suggests that many users have multiple email identifiers, which could be
considered as a proxy for identity if tied to service accounts. Similarly, data
suggests that a large minority of Twitter users have multiple Twitter
accounts~\cite{twitter}.

\paragraph{Multiple browsers}

Using multiple browsers is a useful technique for managing multiple accounts.
Some services support multiple accounts but not multiple login (e.g.,
Twitter). For services that do (e.g., many Google services), the user interface
may be so confusing that it leads to errors, so the best recommendation may be
to use multiple browsers~\cite{yegge}.\footnote{Using multiple
browsers as an alternative to multiple login:
\url{https://support.google.com/accounts/bin/answer.py?hl=en&answer=179235}}
For users who want to separate work and
personal browsing, using multiple browsers is the easiest solution.

\paragraph{Multiple devices}

Many users have multiple devices and many use them for different purposes.
Sometimes the difference in purpose is due to the nature of the device (e.g.,
GPS and mapping software are probably more often used in mobile devices) and
some may be due to policy (e.g., personal browsing is limited to a personal
laptop). Using multiple devices implicitly creates multiple browser states and
thus multiple identities.  However, these identities may still be linked
through authentication data or other techniques.%~\cite{aol}

\paragraph{Multiple profiles}

Several browsers support separation of profile data, including cookies,
history, passwords, and other local storage. Firefox 
%\url{https://bugzilla.mozilla.org/show\_bug.cgi?id=214675\#c53}.}
and Chrome support multiple profiles, though neither of these is very
discoverable or easy to use.\footnote{Chrome:
\url{https://support.google.com/chrome/bin/answer.py?hl=en&answer=2364824},
Firefox: \url{https://bugzilla.mozilla.org/show\_bug.cgi?id=214675\#c53}} The
threat model does not include users with malicious intent in the same
household.

\paragraph{Private browsing mode}
Private browsing mode exists in all major browsers, but has no standard
behavior~\cite{ABBJ10}.
Interaction with extensions, treatment of cookies, history, and
bookmarks upon entering and exiting are different across browsers.
Similar to multiple profiles, private browsing mode is not secure against local
attacks.

\paragraph{Cookie blockers}
Disconnect and ShareMeNot are browser extensions to disallow interacting with
service providers like Google and Facebook unless users explicitly choose to
interact with that site~\cite{disconnect,franzi}.  Collusion is a tool for
visualizing and blocking cookies~\cite{collusion}, in particular third-party
cookies set by tracking sites which are typically ad networks.

\subsubsection{Linking identities}

For many users, the natural separation that occurs when creating multiple
accounts, using multiple browsers and devices is a drawback, not a benefit.
These users want fewer contextual identities and use the following techniques
to merge them.

\paragraph{Building social graphs}

Users can explicitly create links between contextual identities, e.g.  linking
to their blog from their Flickr profile, or resharing a blog post via Twitter.
It is also easy to implicitly create links between contextual identities,
sometimes accidentally: generating two isomorphic graphs at different services
is often enough to re-identify an individual~\cite{narayanan1}. The plethora of
cross-posting software available suggests that users often want to link
identities.

Building reputation online is a long and arduous process; a user who has built
up high credibility in one social network may want to transfer that credibility
when using a new service by using the same identifier or somehow proving in the
new service that they control other credible identities.  The advent
of verified account mechanisms in Twitter, Google Plus and Facebook indicates
that this problem is on the rise.

\paragraph{Synchronizing browser data}

All five major browsers (Firefox, Chrome, Safari, Internet Explorer, and Opera)
offer synchronizing some set of browser data across devices, with varying
degrees of success. Multiple cross-browser applications also exist to
synchronize bookmarks and passwords.\footnote{Bookmark sync: \url{xmarks.com}, password sync: \url{lastpass.com}}

\subsubsection{Curating identities}

Given the large number of social network users and that 30-40\% of spoken
communication is devoted to informing others about ourselves, many users are
bound to share information they later regret~\cite{tamir}.

\paragraph{Service settings} Service configurations such as Facebook friends,
Google circles and privacy settings are intended for users to self-manage
privacy and identity. However, shifting privacy settings and complex,
interacting features make these configurations unpredictable for many users.

\paragraph{Auditing} Wang et. al suggest that common techniques
for handling regret online include manual deletion of regret- table posts,
self-censoring or delaying posts that they predict might bring
regret~\cite{wang}. Pew Internet reports that 57\% of Internet users
periodically search for themselves to manage reputation, with young adults
being the most active~\cite{pew3}.

\subsection{How can we improve identity management tools?}

Service providers are highly incentivized to make linking identities easy;
more tools to help separate and curate identities are necessary. Without
knowing more about how users think about identity, we concentrate solely on
automating manual processes that users already perform.

\paragraph{Mitigating accidental linkage} Federated login services can prevent
linking contextual identities to the same user identifier. For example, a
federated login service that supported multiple identifiers could prevent the
user from associating the same identifier to radically different
contextual identities, such as ones for dating and professional use.

\paragraph{Auditing}
A browser is in an ideal position to intermediate social network
posts, aggregate them locally, and present them to the user when they desire to
audit their digital footprint.  For example, a user could see all of the
comments and posts they made in the last week and redact content that is
sensitive or too negative, a frequent cause of regret~\cite{wang}. 
Such a tool could incorporate sentiment analysis to aid the user in finding
particularly problematic content.

\paragraph{Expiring posts}
Humans don't have perfect memories, and neither should social
networks~\cite{delete}. Many users already engage in manual auditing and
deletion of old posts~\cite{fbtips2}. A better solution would be to build tools
that let the user manage this more easily through an API: both Twitter and
Facebook provide APIs for post deletion.
%\footnote{Google Plus offers a read-only API.}

\paragraph{Expiring contextual identities}
Engaging in a long-term task such as house-buying often requires the
construction of a contextual identity. The tasks associated with this identity
have an externally imposed end date and may have long-lived side-effects, such
as long-term cookies that reveal sensitive information (e.g., the neighborhood
of the house purchase, income information derived from purchases, other
demographic data). For contextual identities that have outlived their
usefulness, users would benefit from destruction of side-effects such as cookie
data and service accounts.

\section{Summary}
People have multiple contextual identities. Sharing personal information in
appropriate contexts fosters positive human interaction.  We hope that
contextual identity serves as a useful notion for developers to understand
their users' privacy needs and develop tools to help users manage their
identities.

\begin{comment}
\section{Acknowledgements}
The authors thank
Lucas Adamski
Ben Adida
Mike Connor
Chris Karlof
\end{comment}

\bibliographystyle{plain}
\bibliography{paper}
\end{document}
